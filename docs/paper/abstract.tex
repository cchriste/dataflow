\begin{abstract}

As dataset sizes grow, data analysis tasks in High Performance Computing
increasingly depend on sophisticated data-flows and out-of-core methods for
efficient system utilization. In addition, as HPC systems get larger, memory
access and data sharing are becoming performance bottlenecks. Cloud computing
is a data processing paradigm typically built on a loosely connected group of
low cost computing nodes without shared storage or memory. An emerging
exemplar in cloud computing is Apache Spark. In a sense, larger HPC systems
may start to look like very high performance clouds, and therefore we wish to
consider the potential of utilizing cloud frameworks in the context of HPC
data processing.

Our work proceeds by identifying common parallel analysis data flows for both
MPI-based and cloud-based applications, such as map and reduce. We construct
and perform several microbenchmarks to determine the performance
characteristic of these tasks using both types of system. In addition, we
compare the results for two real-world analysis tasks. The results of our
experiments are discussed in the context of their applicability to future HPC
architectures. Beyond understanding performance, our work is a demonstration that
technologies such as Apache Spark, while typically aimed at multi-tenant
cloud-based environments, can be used effectively in a traditional
clustering/supercomputing environment via the job scheduler. In the
case of Apache Spark, we also consider whether language plays a role,
especially when writing code in Python or Java/Scala.

\end{abstract}

% TODO: Get ACM categories right.
% A category with the (minimum) three required fields
\category{H.4}{Information Systems Applications}{Miscellaneous}
%A category including the fourth, optional field follows...
\category{D.2.8}{Software Engineering}{Metrics}[complexity measures, performance measures]

\terms{Theory}

\keywords{ACM proceedings, \LaTeX, text tagging}
